\textbf{Introduction}

Legal positivism is, in the first instance, a thesis about the
conceptual foundations of legal validity.\footnote{\emph{See},
  \emph{e.g.}, Mark Greenberg, \emph{How Facts Make Law}, \textsc{Legal
  Theory} 10(3)(2004)157:198; John Gardner, \emph{Legal Positivism: 5 ½
  Myths}, A\textsc{m. J. Juris}., 46(1)(2001): 199--227; Christopher
  Kutz, \emph{The Judicial Community}, \textsc{Phil. Issues}
  11(1)(2001): 442-469; David Enoch, \emph{Reason-Giving and the Law},
  in Leslie Green and Brian Leiter, eds., \textsc{Oxford Studies In
  Philosophy of Law}, Volume 1 (Oxford: Oxford University Press, 2011):
  1-15.} It seeks to parse the \emph{social} mechanisms that underlie
the \emph{legal} decisions that, together, determine the substance and
govern the processes of a given legal system.\footnote{\emph{See} Gerald
  J. Postema. \emph{Coordination and Convention at the Foundations of
  Law}, JOURNAL OF LEGAL STUDIES, 11 (1982): 165--203, at 167 (``Thus
  positivists in the tradition stemming from Bentham locate the
  {[}criteria of validity{]} in matters of social fact, thereby
  rejecting the view that the validity of a law is a function of its
  truth or moral soundness.''); \emph{see also} Andrei Marmor, \emph{The
  Rule of Law and its Limits}, \textsc{Law and Philosophy}, 23(1)(2004):
  5.} Natural law approaches, by contrast, seek to parse not only the
social but also the \emph{moral} mechanisms that underlie legal
determinations of validity.\footnote{\emph{See} Kutz, \emph{supra} note
  1, at 444-45. John Finnis, \emph{On the Incoherence of Legal
  Positivism}, 75 \textsc{Notre Dame L. Rev.} (2000): 1597-98.} For both
positivists and natural lawyers, proposition \emph{P} accurately states
the law governing some topic in a given jurisdiction, if and only if
\emph{P} meets the applicable criteria of legal validity, \emph{C}, in
that jurisdiction.\footnote{Kutz, \emph{supra} note 1, at 443.} The
fundamental difference between the two approaches is that positivists
maintain that \emph{C} obtains in only virtue of social mechanisms,
while natural lawyers remain open to the possibility that \emph{C}
obtains in virtue of moral mechanisms.

For positivists, the social mechanisms at the foundations at the
foundations of the law are explicated through \emph{descriptive} facts:
claims about the behaviors or mental states of judges, legislators,
executives, or voters.\footnote{Greenberg, \emph{supra} note 1, at 157.
  For the classic statement, \emph{see} H.L.A. Hart, \emph{Positivism
  and the Separation of Law and Morals} 71 \textsc{Harv. L. Rev.}
  71(4)(1958): 593-629. As Ronald Dworkin put it, for positivists, legal
  validity is a matter ``not {[}of the{]} content {[}of norms{]} but
  with their \emph{pedigree} or the manner in which they were adopted or
  developed.'' Ronald Dworkin, \textsc{Taking Rights Seriously}
  (Cambridge: Harvard University Press, 1978): 17.} So, for instance,
positivists may point to facts about the behavior of courts to explain
the content of \emph{C}, but not to the merits of \emph{C} itself.
Positivists, in other words, reject grounding their explanations of the
criteria of legal validity in \emph{value} facts: normative or
evaluative claims.\footnote{Greenberg, \emph{supra} note 1, at 157.
  There is, of course, the matter of hard and soft positivism. Soft
  positivists allow that value facts can possess derivative significance
  in determining the content of the law. Hard positivists argue that
  value facts can never do even that. The debate over whether value
  facts can affect the content of laws, not whether they can affect the
  criteria of legal validity.} These are taken to bear only an arbitrary
relationship with legal norms.\footnote{\emph{Id}. at 157-59.
  Positivism's insistence on the primacy of descriptive facts gives rise
  to a thorny question: How can facts about what we happen to have been
  doing in the past give rise to genuine reasons about what we should be
  doing in the future? Or, put differently, how can it be that what we
  happen to do around here can provide reasons to keep doing it?
  \emph{See}, \emph{e.g.}, Kutz, \emph{supra} note 1, at 446.} Natural
lawyers, by contrast, might well explain the content of \emph{C} through
reference to value facts. Perhaps, for instance, \emph{C} states the
criteria of legal validity because it is demanded by democratic values.
Explanations of these sorts are not open to positivists, for whom
descriptive facts possess explanatory primacy. In giving descriptive
facts explanatory primacy, positivists are committed to the claim that
there is no law, properly so called, absent that which is created
through social mechanisms.

The most basic of these social mechanisms -- and the focus of our
attention in this paper -- is the process by which legal officials posit
a law, and legal subjects respond. Say that a legislature duly enacts
proposition \emph{P}. What follows? What is the characteristically
\emph{legal} way in which subjects are influenced by \emph{P}? The way
in which one answers that question determines her conception of
\emph{legal guidance}.

A conception of legal guidance explains how propositions of law
influence individual conduct. Although positivism is, in the first
instance, a thesis about legal validity, it is not only that. It is also
a descriptive theory that seeks to explain how law, ``as something that
must be \emph{posited} through some social act or activity, either by
enactment, decision, or practice,''\footnote{Scott Shapiro, \emph{Law,
  Morality, and the Guidance of Legal Conduct}, \textsc{Legal Theory}
  6(2)(2000): 127 (emphasis in original).} is able to influence
individuals' beliefs and actions.

This question -- explaining how, exactly, it is that a duly-enacted law
influences individuals -- is particularly pressing for legal
positivists. If a natural lawyer claims that \emph{value} facts possess
explanatory primacy in an account of legal validity, such that, at least
in some cases, when \emph{P} requires one to ϕ, one \emph{thereby} has a
genuine or unqualified reason to ϕ, then the task of explaining the
connection between law and practical reasoning is, arguably, less
urgent. For natural lawyers, the connection between the content of
\emph{P} and individual beliefs and action is more straightforward. In
short, if \emph{P} is underwritten by some moral logic, it might more
readily gain a foothold in the beliefs and actions of legal
subjects.\footnote{There are a number of reasons why this might be so.
  For instance, one might think that reasons that are normatively
  desirable tend also to motivate individuals. On such an account, one
  might hold that, if an agent engages in rational deliberation, she
  will arrive at the morally correct conclusions, whatever her original
  motivations and desires. For a classic treatment, \emph{see} Bernard
  Williams, \emph{Internal and External Reasons}, in \textsc{Moral Luck}
  (Cambridge: Cambridge University Press) (1979): 101-13. We should also
  observe that, even for a natural lawyer, many laws will not possess an
  intrinsic connection with morality. Natural law is not committed to
  the idea that \emph{every} proposition of law is connected to
  morality; only that \emph{some} might be. For those that are not,
  natural lawyers face roughly the same question that positivists do.}
Positivists, however, have nothing of the sort upon which to rely. They
must, instead, explain how it is that laws, which are not necessarily
underwritten by any normative logic, guide individuals' reasoning and
behavior.

The centrality of legal guidance has not eluded positivists. Scholars
working in the field have developed a comprehensive conception of legal
guidance and set out the social and legal conditions under which it
obtains. The conventional wisdom conceptualizes legal guidance as that
state of affairs in which a subject ``is able to learn of his
obligations or rights from {[}the relevant legal norm{]} without
engaging in deliberation.''\footnote{\emph{See} Shapiro, \emph{supra}
  note 8, at 153.} This conception of legal guidance treats guidance as
a purely \emph{epistemic} phenomenon. Law guides when it possesses the
capacity to \emph{convey information} about the content of the law to
subjects. In order to possess this capacity, positivists hold, ``there
are certain conditions that the law has to meet.''\footnote{Marmor,
  \emph{supra} note 2, at 5.} A wide consensus holds that eight
attributes ``state necessary conditions'' \footnote{Scott Shapiro,
  \textsc{Legality} (Cambridge: Harvard University Press): 395. Earlier,
  Shapiro observes that ``regimes that flout these principles are simply
  not engaged in the basic activity of law.'' \emph{Id}. at 394.} for
the existence of legal guidance and thus legality. These attributes are,
by now, familiar. Laws must be general; they must be promulgated; laws
must not succumb to retroactivity nor contradiction; they must be clear,
and possible to follow; there must be congruence between the rules as
written and as applied; and, finally, laws must be stable. That legal
norms must possess each of these attributes in order to guide -- in the
sense of conveying to subjects their legal rights and obligations --
constitutes one of the core claims of analytic jurisprudence.\footnote{\emph{See
  generally} H.L.A. Hart, \textsc{The Concept of Law} (2)(Oxford: Oxford
  University Press, 1997); Joseph Raz, \emph{The Rule of Law and its
  Virtue}, in \textsc{The Authority of Law} (1979); Jeremy Waldron,
  \emph{The Concept and the Rule of Law}, \textsc{Harv. J. L. \& Pub.
  Pol'y.} 30(2006):15-30; John Gardner, \emph{Hart on Legality, Justice,
  and Morality}, \textsc{Jurisprudence} 1(2)(2010): 253-265.}

This paper challenges the prevailing conception of legal guidance.

\textbf{Epistemic Legal Guidance: Mediating between Law and
Deliberation}

When the U.S. federal government enacts new criminal laws, or seeks to
change health care policy, or changes the rules governing securities
laws, it seeks to regulate the behavior of the over 240 million adults
living within its borders.\footnote{\emph{See} Shapiro, supra note 12,
  at 72 (describing ``the regulation of mass populations''). \emph{See
  also} Jeremy Waldron, \emph{The Rule of Law in Contemporary Liberal
  Theory}, 2(1) \textsc{Ratio Juris} (1989): 79-96. The basic
  demographic contours of the U.S. population are available at United
  States Census Bureau,
  \href{../customXml/item1.xml}{http://quickfacts.census.gov/qfd/states/00000.html}.
  It should be noted that U.S. law, in addition to governing those
  within the territorial United States, also applies to certain
  individuals outside its borders.} Regulation at what is essentially a
global scale poses a problem: how is the state to \emph{inform} this
number of individuals of their new duties, rights, or obligations? Even
if each individual governed by a new regulation could be located, it is
clearly beyond the capacity of the state to send officials door-to-door
providing updates on new regulations.\footnote{Although, on certain
  occasions, it does do things like community workshops to help inform
  citizens of their new duties, rights, and obligations.} As a
substitute, the state crafts legal norms that provide citizens with
\emph{epistemic guidance}. Laws provide epistemic guidance when a
citizen ``is able to learn of his obligations or rights from {[}the
relevant legal norm{]} without engaging in deliberation.''\footnote{\emph{See}
  Shapiro, \emph{supra} note 8, at 153.} More colloquially, laws provide
epistemic guidance to individuals when an individual is able to read the
law and thereby learn how to act in compliance with the law. Such
\emph{impersonal} guidance is a substitute for the \emph{personal}
guidance rendered infeasible by the size of the modern state and its
large-scale regulatory schemes.\footnote{On occasion, law is used, of
  course, to regulate single individuals or small groups. Legislatures
  pass bills aimed at single individuals. \emph{See}, \emph{e.g.},
  Jeffrey S. Hill and Kenneth C. Williams, \emph{The Decline of Private
  Bills: Resource Allocation, Credit Claiming, and the Decision to
  Delegate}, 37(4) \textsc{Am. J. Pol. Sci. (1993): 1008-1031.} Judicial
  decisions affect only the parties to the case, in the first instance.
  And administrative agencies adjudicate claims of single persons.}

Epistemic guidance offers a solution to the problem of regulating
large-scale populations. But not all laws provide epistemic guidance. A
wide consensus holds among legal positivists that epistemic guidance is
possible \emph{only} if the legal norms that a state promulgates meet
certain parameters. In order to guide epistemically, positivists hold,
``there are certain conditions that the law has to meet.''\footnote{Marmor,
  \emph{supra} note 2, at 5.} Let us call these conditions -- which
positivists argue ``state necessary conditions'' \footnote{Scott
  Shapiro, \emph{supra} note 12, at 395. Earlier, Shapiro observes that
  ``regimes that flout these principles are simply not engaged in the
  basic activity of law.'' \emph{Id}. at 394.} for the existence of
epistemic guidance -- the \emph{attributes of guidance}.

Conventional wisdom among positivist legal philosophers holds that, in
order to guide individual conduct, ``there are certain conditions that
the law has to meet.''\footnote{Marmor, \emph{supra} note 2, at 7.} A
wide consensus holds that there are eight attributes that ``state
necessary conditions for the existence of {[}law{]}.''\footnote{Shapiro,
  \emph{supra} note 12, at 395.} We need only briefly review the
substance of these conditions. Laws must be (1) \emph{general}, setting
forth rules of conduct, applicable to some defined segment of the
population, prohibiting or facilitating certain modes of behavior. These
rules must be published, or (2) \emph{promulgated}, so that they are
available to those whose conduct they govern on a (3)
\emph{prospective}, rather retrospective, basis. Laws must avoid
obscurity or unintelligibility; they must be (4) \emph{clear}. A legal
code must be (5) \emph{non-contradictory}, in that provisions do not
conflict. Closely related is the demand that law be (6) \emph{possible}
to follow. Official enforcement of these laws should be (7)
\emph{congruent} with the laws as written. And, finally, laws should be
(8) \emph{stable}, which requires that norms not change too frequently.
As a knife must be sharp if it is to cut well, so too must laws meet
these criteria if they are to provide epistemic guidance.\footnote{Raz
  helpfully compares law to a knife, in the sense that both can possess
  certain properties that render them, as tools deployed by individuals,
  more or less useful. Our question can usefully be framed: If sharp
  knives cut well, what sort of laws govern conduct well? \emph{See}
  Raz, \emph{supra} note 13, at 225.}

We, of course, are concerned here with the attribute of stability in
particular. Ordinarily one might pause here and explicate the idea of
stability in some detail. The difficulty is that positivists have
remained rather coy about just what stability amounts to. Andrei Marmor
captures contemporary sentiment about stability: ``This requirement of
the rule of law is basically a rough standard,'' argues Marmor, because
``it would be absurd to assume that we can have a precise notion of the
ideal pace of change.''\footnote{Marmor, \emph{supra} note 2\textsc{,}
  at 34. It is instructive to observe that Marmor suggests that, in
  discussing stability, we are searching for an ``ideal'' pace of
  change. This interpretation of the attributes of guidance seems overly
  moralized. The items on the list explain how laws should be crafted if
  they are to guide citizens; they do not instruct officials how to make
  ideal policy.} Our inability to formalize the stability constraint
seems to stem from the fact that, unlike other attributes of guidance,
it is difficult, if not outright impossible to define stability in the
abstract. To characterize a law as stable seems, in some fundamental
way, to involve an assessment of the conditions in a society, and the
expectations of the governed, in a way that the other items on the list
do not. In the second half of this paper I will offer a conception of
motivational guidance that cashes out stability in just these terms --
but prior to reaching that point, we will need to investigate the
conventional approach to stability.

The conventional wisdom holds that, if a legal norm meets the criteria
stated by the attributes of guidance, then it provides epistemic
guidance and, thus, is capable of \emph{making a difference} to
individuals' beliefs and conduct -- ``a difference, that is, in the
structure or content of deliberation and action.''\footnote{Jules
  Coleman, \emph{Incorporationism, Conventionality, and the Practical
  Difference Thesis}, 4(4) \textsc{Legal Theory (1998): 383.} We can
  test whether or not an individual was guided by a rule by asking about
  the counterfactual case: A legal rule, \emph{P}, guides a person to do
  some act, \emph{A}, only if that person ``might not have done''
  \emph{A} ``had he not appealed to'' \emph{P} qua legal rule.
  \emph{See} Shapiro, \emph{supra} note 8, at 132.} Epistemic legal
guidance, then, and the attributes of guidance that make it possible,
provide a crucial explanatory link for positivists. It explains how
laws, which begin as strings of characters posited by officials, govern
society.

Crucial to this explanatory chain is the idea that individuals, once
they learn of their rights and obligations from law, take the content of
law as reasons for action.\footnote{Enoch, \emph{supra} note 1; Gerald
  J. Postema, \emph{Implicit law}, LAW AND PHILOSOPHY, 13(3)(1994):
  361--387; Stephen Perry, \emph{Hart's Methodological Positivism}, in
  HART'S POSTSCRIPT: ESSAYS ON THE POSTSCRIPT TO THE CONCEPT OF LAW
  (Oxford: Oxford University Press, 2001).} Law transmits its directives
by providing individuals with things called reasons.\footnote{Pamela
  Hieronymi, \emph{Reasons for Action}, \textsc{Proceedings of the
  Aristotelian Society}, 111(2011): 407--427.} Individuals, in turn, are
equipped to process and make sense of these reasons. Although
philosophers dispute the exact set of processes that individuals use to
process reasons -- is reasoning machine-like or driven in part by
emotions? -- it is generally agreed that the processing of reasons
involves some, most likely imperfect, process of \emph{weighing}.
Discrete considerations can be measured against each other, and added
and subtracted, in order to help individuals arrive at this or that
belief. Hence we often speak of a \emph{balance} of reasons. Legal norms
provide epistemic guidance when they possess the capacity to add a
reason to one side or the other of the balance.

At this point one might well wonder, when we speak of ``legal norms,''
if we are referring to all legal norms, or only a subclass of them. One
could, following Joseph Raz, argue that only the legal norms governing
the \emph{processes} that produce legal rules must be stable, not the
legal rules themselves.\footnote{Raz, \emph{supra} note 13, at 214-216.}
Guidance, Raz argues, requires merely that ``the making of particular
laws should be guided by open and relatively stable general
rules.''\footnote{\emph{Id}. at 215.} What distinguishes a general rule?
Raz specifies that ``two kinds of general rules create the framework for
the enactment of particular laws,'' namely, ``those which confer the
necessary powers for making valid orders'' and ``those which impose
duties instructing the power-holders how to exercise their
powers.''\footnote{\emph{Id}.} Under Raz's framework, general rules
govern the creation of the primary rules that regulate individual
conduct. And it is general rules that must be stable in order that law
guide conduct.

Notice, however, that it is also the case that any given process rule
can be described as a discrete rule that applies to some legal official.
\footnote{U.S. Senate rules governing quorums seem like an exemplar of
  Raz's power-conferring rules. According to Senate rule 6.1, ''A quorum
  shall consist of a majority of the Senators duly chosen and sworn.''
  So far, the rule governing quorums does not seem to apply to
  individual conduct. However, if we glance up at Senate rule 4.1a, we
  find that rule governing quorums directs the Presiding Officer on the
  matter of when to commence daily sessions.} All process rules are
discrete rules, in this sense. Thus, any argument we can run to show
that stability is not a necessary condition for epistemic guidance works
for both discrete norms and process rules. Insofar as stability is
concerned, Raz's distinction doesn't make a difference.\footnote{Raz's
  reason for drawing this distinction is well taken. Raz is concerned
  that if we do not draw the distinction, most of administrative law
  will fail to be counted as law, since it is notoriously volatile. Raz
  singles out administrative regulations as the site at which stability
  of general legal processes are of particular import for the stability
  requirement. We will show later in the argument that one can sustain
  administrative law as law without resorting to the discrete rules
  versus process rules distinction.}

Moreover, instability of this sort -- instability so pervasive that it
affects the conventions sustaining the creation of laws -- is the type
of instability that undermines an entire legal system. Legal systems
tend not to survive such dramatic shake-ups. And legal systems, as such,
are standardly defined as stable entities.\footnote{Shapiro,
  \emph{supra} note 12, at 65.} If Raz's distinction \emph{does} make a
difference, it proves too much. Thus, absent a compelling reason to
distinguish types of legal norms, we shall use the term to refer
indiscriminately to both general and primary rules.

It \emph{is} important, however, to distinguish between legal and
non-legal norms. Unsurprisingly, legal norms do not monopolize
individuals' deliberations. \emph{Non-legal} reasons compete with legal
reasons for individuals' allegiance. In contemporary states, individuals
receive guidance not only from legal norms, but also from customary
norms, ethical norms, religious norms, and social norms. Each of these
types of norms might \emph{also} purport to regulate individual conduct.

The state assists individuals in locating the \emph{legal} norm by
adorning it with ``an official marking that designates certain standards
as those to which one must conform.''\footnote{Jules L. Coleman,
  \textsc{The Practice of Principle: In Defense of a Pragmatist}

  \textsc{Approach to Legal Theory} (Oxford: Oxford University Press,
  2003): 137.} Common marks include ``inscription in some authoritative
text'' or ``declaration by some official.''\footnote{Scott Shapiro,
  \emph{What is the Rule of Recognition (and Does it Exist?)},
  \textsc{Yale Law School Public Law and Legal Theory Research Paper
  Series} (unpublished manuscript): 2, available at
  \emph{http://papers.ssrn.com/sol3/papers.cfm?abstract\_id=1304645}.}
Official marks allow citizens to identify that a norm is, in fact, a
legal norm without having to engage in deep deliberation and speculation
as to the pedigree of the norm. For positivists, a legal norm is valid
just in virtue of certain social facts, usually certain political
procedures, such as passage of bills, and the like. Adorning legal norms
with official marks is designed to distinguish legal norms from other
sorts of norms, and to ensure that legal subjects need not engage in the
deep historical and conceptual tasks that would be involved in an
investigation of whether a given norm is, in fact, a legal
norm.\footnote{\emph{See} Hart, \emph{supra} note 13, at 95.} Official
marks attempt to solve the informational problem that arises when
competing norms purporting to regulate proliferate. An official legal
mark conveys to the individual the idea, as far as the law is concerned,
\emph{this} norm is dispositive.\footnote{Nonetheless, an individual
  might wonder why comply with the legal norm rather than some other
  norm. Official markers do not provide an all-things-considered reason
  to comply with the legal standards, but, in designating certain
  standards as those with which law requires conformity, official
  markers convey that legal reasons do not take themselves as subject to
  the balance of total reasons. Official markers demonstrate that legal
  reasons, simply in virtue of their status, not only are not subject to
  balance of all reasons, but take themselves to have changed the
  balance of reasons in a Razian manner or do not take themselves to
  have changed the balance of reasons but do simply disregard the other
  reasons.}

Epistemic guidance, therefore, involves \emph{presenting} individuals
with the legal norm alongside whatever non-legal norms to which they
adhere. In other words, epistemic guidance models a \emph{thin}
integration between the reasons created by legal norms and an
individual's deliberative processes. Individuals pursue their
antecedently chosen ends, and, in the pursuit of those ends, encounter
legal norms. Background beliefs and desires determine whether, and to
what extent, individuals comply with legal norms. Legal norms provide
something of a map of possible routes for individuals, assuming that the
individual knows her destination. Which route any given agent takes is a
function of antecedently given desires. And the individual may choose to
simply not consult the legal map at all, depending on these antecedently
given desires. Epistemic guidance does not require that individuals
choose their ends because of legal norms -- only that individuals can
make themselves aware of what the law requires.\footnote{Here is another
  way of putting the point. Assume an individual with a set of beliefs,
  B, about what the law requires. Artificially separate the individual's
  desires into some set of desires toward compliance the law, D, and all
  of her other desires, O. Individuals' deliberations concerning the
  pursuit of O are constrained by the interaction of B and D. So, e.g.,
  if the individual wants to get rich, but also wants to comply with the
  law, then things like robbing a bank will be ruled out. Now, in the
  ordinary course of things robbing a bank will also be ruled out by
  moral or social desires. In such a case, that the individual doesn't
  rob a bank is overdetermined. But let's assume that, for this
  individual, moral or social pressure doesn't rule out robbing a bank;
  the desire to comply with law is the only constraint. Say B changes,
  to B-prime. Now, O is constrained by B-prime. O does not change; only
  the content of the constraint is changed.} This would seem to be the
case even on a Razian account premised on peremptory reasons.\footnote{\emph{See}
  Joseph Raz, \emph{Authority, Law and Morality}, 68(3) \textsc{The
  Monist} (1985): 295-324. I am very grateful to an anonymous reviewer
  for raising this objection.} For even if it is the case that law
presents itself as a practical authority, and is ``meant to replace the
reasons on which it depends,'' it still may be the case that the citizen
determines that the reputed practical authority is no authority at
all.\footnote{\emph{Id}. at 297.} By \emph{the rules of the legal game},
legal norms purport to be practical authorities; but epistemic guidance
does not require that citizens themselves conclude, as an all-in
judgment, that legal rules are always authoritative.\footnote{As Raz
  writes, ``No blind obedience to authority is here implied. Acceptance
  of authority has to be justified \ldots{}'' \emph{Id}. at 299.
  Epistemic guidance, for Raz, requires that law ``must be capable of
  guiding the behavior of its subjects. It must be such that they can
  find out what it is and act on it.'' Raz, \emph{supra} note 13, at
  214.} If it did, it would imply that epistemic guidance is only
satisfied when all citizens comply all the time. All that epistemic
guidance requires is that those that the legal norms be written in such
a way as to convey their normative demands to citizens.

The conventional wisdom holds that epistemic guidance is necessary in
order that individuals be made aware about what the law requires of
them. In a world in which personal guidance is infeasible, individuals
learn of their rights and obligations impersonally, through epistemic
guidance. Because individuals' beliefs and actions are determined by
weighing reasons, legal norms affect beliefs and action by creating new
reasons for individuals. Thus, on this descriptive model, legal norms
operate via ``self-directed action.''\footnote{Postema, \emph{supra}
  note 25, at 369.} In other words, individuals must apply general legal
directives to their own particular situations, thus overcoming the
problem of mass regulation. Law thus guides by seeking to affect
individuals with reasons, rather than, with say, unadorned brute force.

\textbf{Does Epistemic Guidance Require that Law be Stable?}

In order to guide by reason -- in order to provide the epistemic
guidance crucial for governing mass populations -- must legal norms, in
fact, meet the criteria set forth by the attributes of guidance? Given
this picture of epistemic guidance, which mediates between laws on the
books and individuals deliberations actions, is it, in fact, true that
the attributes of guidance state necessary conditions for epistemic
guidance? More formally, we may ask of any of the eight attributes:

\begin{quote}
What state of the world does epistemic guidance require such that some
attribute, \emph{X}, is necessary, assuming the operation of some other
set of attributes, \emph{Y}, where \emph{X} ∉ \emph{Y }
\end{quote}

We answer this question by posing a counterfactual. An attribute is
necessary for a conception of legal guidance when the absence of that
attribute renders law incapable in principle of guiding individuals. The
idea here is that there are certain states of the world that must obtain
in order that legal guidance be possible, and the items constituting the
attributes of guidance either uniquely cause those states or are
necessary for those states to exist. Posing a counterfactual in which
legal norms do not possess the attribute in question allows us to test
whether guidance is possible in its absence. Applying this framework to
the case of stability, we can generate two cases: one in which a legal
norm is stable and a counterfactual case in which a legal norm is
unstable. We then compare how these two cases affect epistemic guidance.

First, imagine a successful case of epistemic guidance, in which the
applicable law, \emph{P}, is stable. Alice desires a certain end, and,
prior to consulting the applicable law, is indifferent between two
available means to the end. Because she is motivated to comply with law,
Alice seeks epistemic guidance. Alice first seeks indirect guidance --
from her associates, secondary sources, and so forth -- and receives a
uniform set of reports as to the content of the law. Just to be sure,
Alice reads the statute and applicable judicial opinions, and finds that
her indirect guidance is supported by the direct guidance. One
alternative is within the bounds of compliance, and the other
alternative is prohibited. Alice chooses the legal alternative,
\emph{P}, and complies.

Our successful case of epistemic guidance provides a baseline against
which we can pose a counterfactual. Can we alter the case such that
instability interferes the transmission of legal content, thereby
impairing epistemic guidance? It is difficult to imagine such a case.
Even if the contents of \emph{P} changed at random intervals, Alice
could, in principle, locate the content of the law and thus receive
epistemic guidance. This conclusion follows from the definition of
epistemic guidance. No matter how volatile a discrete legal rule, if it
is duly enacted, then it bears law's official mark. And the fact that a
norm bears law's mark implies that the informational problem about which
norm is authoritative has been solved, and epistemic guidance is, in
principle, possible. Duly-enacted laws are published, of course, thanks
to the promulgation requirement. Laws altered in any random interval are
still subject to promulgation, and the fact of promulgation ensures that
individuals can in principle locate the content of the law, which is all
the epistemic guidance requires. Epistemic guidance simply does not
require any state of the world for which stability is required. Contra
the conventional wisdom, then, it seems that no level of instability
interferes with the capacity of legal norms to provide epistemic
guidance.

\textbf{Accounting for the Error in the Conventional Wisdom }

If our account thus far is correct, the conventional wisdom, which holds
that stability is a necessary attribute of epistemic guidance, is
incorrect. Where did those accounts go wrong? We need an error theory.
Here, we will explore why it is commonly believed that epistemic
guidance requires stability. The culprit is that instability raises the
costs of epistemic guidance. Instability makes it more difficult to
locate the law, thus making it more difficult to comply with the law. On
this line of thought, instability sets off a chain reaction that ends
with decreased compliance.

This line of thought dovetails with the most common way to justify
stability as an attribute of guidance. As earlier noted, the most common
way to do so is on grounds that, if law is volatile, individuals will
not rely on it. This claim, or some version of it, might well be true.
The problem with this argument is that epistemic guidance does not
require increasing individuals' reliance on law or otherwise providing
motivational guidance. It does not require legal officials to cultivate
reliance or that individuals be disposed to rely. If the attributes of
guidance \emph{are} conceptualized as attributes that tend to increase
guidance or compliance, then it would make sense to justify stability on
reliance grounds. The difficulty here is that if we conceptualize the
attributes of guidance in this way, then our list of eight attributes
is, and always was, woefully incomplete. If the standard is attributes
that tend to enhance compliance, then surely Pareto optimality (or some
such) is missing from the list.

Nonetheless, one might still be worried about the possibility that high
volatility makes epistemic guidance more difficult, thus rendering
compliance less likely. This worry is not altogether misguided. Even if
stability is not required for epistemic guidance, it might possess a
contingent, yet predictable, relationship to compliance and thus law's
efficacy.\footnote{Ultimately, the level of compliance sought by legal
  officials is a function of the regulatory regime. Thus, higher or
  lower search costs might be tolerated. We cannot answer that in the
  abstract, so we cannot specify how much non-guidance is compatible
  with a regulatory program. We can only observe that there will be, in
  the normal case, some desired level of guidance and compliance.}
Sketching out such a contingent relationship does help to explain the
error, and explain why we might care about stability even if it is not a
property of legality. The key to sketching out this contingent
relationship between instability and reduced compliance is to treat
legal norms as costly information.

\textbf{Legal Norms as Costly Information}

Ideally, locating legal norms would be frictionless. The environment in
which legal norms operate, however, ensures that locating law's content
is rather complex. In contemporary states, individuals receive guidance
not only from legal norms, but also from customary norms, ethical norms,
religious norms, and social norms. Each of these types of norms purports
to regulate individual conduct. Laws can only claim primacy over other
types of norms if they make themselves known. Earlier, we observed that
official marks \emph{attempt} to solve the informational problem that
arises when competing norms purporting to regulate proliferate. Here, we
show that official marks are \emph{signals} from the state to citizens.
As signals, the information they provide is not always perfect. It is
sometimes noisy, especially when laws are unstable. This situation
raises the costs obtaining legal information, and thus can lead to
reduced compliance.

One way in which individuals might determine if a norm is, in fact, a
legal norm is to engage in an historical inquiry into the pedigree of
the norm. This historical inquiry would involve questions like,
\emph{Did Congress pass this norm}?, \emph{Does the Constitution forbid
norms of this sort?}, \emph{Has any court modified the norm}?, and so
forth. Such investigations, however, are likely to result in confusion.
Instead, we need a proxy or a signal to stand in for all of the
historical investigation that goes into determining if a norm is, in
fact, a legal norm. The best proxy or signal of legality is that the
norm ``bear the mark'' of legal authority. When a legal norm bears the
mark of legal authority, it announces to individuals that it possesses
the proper pedigree.\footnote{One might be tempted to avoid taking the
  term ``mark'' literally, since, at base, to be lawful simply means to
  have the correct pedigree, to be a duly enacted law. But the official
  mark is not equivalent to (or even a logical extension of) the fact
  that a norm is duly enacted. The mark is a proxy designed to solve
  informational problems and decrease deliberation. Only duly enacted
  norms receive official marks, but we can imagine a case in which a
  duly-enacted law does not, for whatever logistical shortcoming, does
  not receive a mark but is nonetheless law. A law that is not published
  would not receive a mark, for instance. If the official mark is to
  solve an informational problem about which norms individuals are
  supposed to comply, then we must take the idea of a mark literally.}
In theory, official marks guide individuals by signaling those norms,
and only those norms, that are sanctioned by the state. As a signal or
proxy, however, it is subject to conveying (1) incomplete or (2)
erroneous information.\footnote{The legal philosophy literature has not
  addressed thse shortcomings with official marks. It has treated the
  idea of official marks quite generously.} In such cases, epistemic
guidance suffers.

Signals can be erroneous because inscriptions in authoritative texts do
not \emph{stop} being signals once the law is superseded. A superseded
statute does not report that it has been superseded. A superseded
statute does not stop bearing the mark of law once it has been
superseded. Ideally, we would devise a technique to signal the content
of the law that could update itself. Official marks as they currently
exist, however, are not self-updating.\footnote{It is worth observing
  that the technology is readily available to make official marks
  self-updating, at least with regard to those official marks published
  electronically. A simple piece of code embedded in the metadata of the
  standing law might well be able to provide self-updating signals.}

One might be tempted to argue that a superseded statute bears an invalid
mark. But marks are not valid or invalid. Rather, they are accurate or
inaccurate, but there is no way of discerning whether they are accurate
or not from the mark itself. Thus, every change in the content of a law
results in the creation of some erroneous signals, unless inaccurate
signals are immediately destroyed. Frequently changing legal norms
results in a proliferation of inaccurate signals. If, for instance, an
official report of the law is printed once every year, and a given legal
norm is changed every year for 10 years, then there will exist nine
superseded reports, unless they are destroyed. These reports are
inaccurate signals of the content of the law.\footnote{Inaccurate, that
  is, assuming that the content of the change is not a reversion to some
  past norm.} Instability thus recreates the informational problem that
official marks were intended to solve. Superseded marks are, in essence,
competitor norms. They state norms that conflict with the state's
position on what counts as compliance with the law.

In addition to the possibility that an official mark provides erroneous
information, official marks might also, or alternatively, provide
\emph{incomplete} information. Although a legal norm contains much of
the information that is necessary to comply with law, it does not always
contain all of the information required for compliance. \emph{That}
information is located in interpretations of the norm, in judicial
opinions, and the like. In other words, an official mark is partial when
it points to a string of text, and refers to that text as law, when, in
fact, the text in question only conveys \emph{part} of the information
an individual needs to know in order to comply with the law. Now, in
common law countries like the United States, almost all official marks
are partial, because much of the meaning of the laws, even statutory
laws, are contained in judicial opinions. Much of the content of highly
abstract laws, like the Eighth Amendment's prohibition on ``cruel and
unusual punishment,'' are contained not in the text of the law but,
rather, in the text of judicial opinions.

In theory, official marks, or signals, point to a string of text, and
tell the citizen: This string of text contains all of the information
you need in order to comply with the law. In practice, however, these
official marks are partial. They convey only some of the information
needed to comply with the law. For instance, in 2000, a prominent U.S.
appellate court lamented a ``familiar'' pattern of activity in which
``Congress passes a broadly worded statute. {[}An administrative{]}
agency follows with regulations containing broad language, open-ended
phrases, ambiguous standards and the like. Then as years pass, the
agency issues circulars or guidance or memoranda, explaining,
interpreting, defining and often expanding the commands in the
regulations.''\footnote{\emph{Appalachian Power Co. v. E.P.A.}, 208 F.3d
  1015, 1020 (D.C. Cir. 2000).} In cases like this, the \emph{ratio} of
information required for a citizen to know how to comply with the law
referenced by the official mark decreases. Such action increases the
total information required to possess knowledge of what counts as
compliance.

If instability increases the erroneousness or incompleteness of official
signals, it increases what we can describe, following Jules Coleman, as
\emph{search costs}.\footnote{Jules Coleman, \textsc{Risks and Wrongs}
  (Oxford: Oxford University Press) \textsc{(2002): 125-131.}} If these
search costs are high enough, the efficacy of a particular legal norm is
reduced. Efficacy is a necessary condition of legal systems, although
not itself an attribute of legal norms. This line of thought does not
show that stability is an attribute of legality, but it may help us to
identify the point at which the costs of finding the law become such
that compliance is compromised to the point that legal norms are
rendered inefficacious.

Volatility can increase the costs of locating the law, thus ramping up
the costs of compliance. Volatility can artificially reduce compliance
by making it costlier than \emph{standing} law would imply that it is.
Assume some ``natural'' rate of compliance with a given law. The rate of
compliance is natural in the sense that it is a function of individuals'
background beliefs and desires. Volatility of the sort we have described
here reduces compliance below this natural rate, even though the content
of the standing norm is the same.

It's possible that instability creates costs sufficiently high that they
deter individuals from locating valid law. Locating valid law in the
face of instability can take time and effort, and can even result in
individuals mistakenly acting on what they believe to be legal reasons.
An example will illustrate how these costs could mount under non-ideal
conditions.

Imagine a variation on Alice's successful case of epistemic guidance.
Assume that Ben, acting in good faith, seeks epistemic guidance. The
content of the applicable legal norm is highly volatile, but Ben is
unaware of this fact. Ben seeks indirect guidance by consulting
secondary sources, which provide mixed reports. Some treatises report
that the law permits only option \emph{P}. Reports from those in a
related industry indicate that \emph{P} and \emph{Q} are permitted.
Sensing that something is amiss (Ben is not a legal official, but is
aware that the law is supposed to be clear and predictable), Ben seeks
direct guidance from what he believes to be an authoritative legal text.
But because the law is volatile, Ben encounters a superseded statute,
which, of course, does not report that it has been superseded. The
statute bears the mark of law, and indicates that both \emph{P} and
\emph{Q} are permissible. Because \emph{Q} is the preferable option all
things equal, Ben adds a consideration to his stack of reasons and acts
accordingly. In fact, Ben erroneously relied on a superseded statute.
The valid statute prohibits \emph{Q}. So Ben fails to comply.

The costs individuals are willing to incur in their search for epistemic
guidance presumably have limits. To be sure, the absence of epistemic
guidance does not imply a lack of compliance in every case. It might
well be the case that most individuals can comply without guidance. But
over the long haul compliance suffers when guidance decreases. And when
volatility increases, compliance decreases because individuals prefer to
save the costs to time and effort. If, at a certain point, the costs
become high enough and the content of the legal norm is non-intuitive
enough that individuals cannot figure out the compliant behavior absent
the norm, compliance could decrease to such a degree that the system can
no longer be described as efficacious.

A second way in which search costs can impair epistemic guidance is when
individuals erroneously rely on a superseded official mark. Here,
individuals desire to comply, and believe that they are complying, but
in fact fail to comply because instability decreases the strength of the
signal sent by officials to subjects. Unlike cases where individuals
decide to stop the search due to the costs of searching and bite the
bullet on compliance, here individuals believe that they have been
guided and are complying.

In our hypothetical, for instance, Ben \emph{took} himself to be relying
on a reason created by law. In fact, he erroneously relied on a
superseded official mark. In such cases, compliance suffers not because
of a decision that is too costly, but, rather, because volatility leads
to a situation in which individuals' \emph{operative} reasons -- the
reasons on which they, in fact, act -- are different from the reasons
that do, in fact, \emph{justify} the action from the perspective of law.
The mistaken non-compliance induced by volatility raises the question of
whether Ben was epistemically guided. We can see that there was no
reason for Ben to act in the way that he did, yet it seems as though Ben
acted for a reason.\footnote{\emph{See, e.g.,} Jonathan Dancy,
  \textsc{Practical Reality} (Oxford: Oxford University Press) 2002.}

\textbf{Three Types of Reasons}

Was Ben epistemically guided when the reason on which he acted was, in
fact, not a reason? By way of analogy, consider a game of chess between
Beginner and Expert.\footnote{Here the account follows Pamela Hieronymi,
  \emph{supra} note 26.} If, say, Beginner is considering whether to
move a bishop vertically, he might consult the rules of chess and find
that bishops are only allowed to move diagonally. The rules of chess
tell Beginner that moving the bishop diagonally (or at least omitting to
move it non-diagonally) is an appropriate action. The rules
\emph{justify} the action.

It turns out that although moving the bishop diagonally was a valid
move, it was not a smart one, and Expert has placed Beginner in a very
precarious situation. Beginner sees that his one way out of Expert's
trap is to undertake what he believes to be a ``castle.'' Beginner
attempts to think through the rules of castling, wondering if he may do
so despite the fact that he had already moved his king. Beginner
mistakenly concludes that the rules of chess do, in fact, allow him to
castle despite already having moved his king. Here, Beginner acts for
what he took to be a reason, but, which, under the rules of chess, is
not actually a reason at all. We can call such reasons \emph{operative}
reasons.

Beginner erroneously believed he could castle after moving the king, due
his lack of familiarity with the rules of chess and a general sense of
anxiety. These reasons \emph{explain} Beginner's choice. By introducing
a distinction between considerations grounded in a normative system
(justificatory reasons) and considerations that an agent took to be
grounded in a normative system (operative reasons), we can see why Ben
was not epistemically guided by the superseded statute.

Instability may create a situation in which individuals predictability
act not on reasons that are not justified. Or, put differently,
volatility may induce individuals to mistakenly believe that they are
being epistemically guided by the legal system when in fact they are
not. Ben, like Beginner, \emph{sought} to comply with the rules. The
non-compliance in both cases resulted from the complexity of the rules
relative to the individuals' capacities to comply. And while we can now
see how to avoid the awkward implication that Ben was in fact guided by
an invalid legal rule, the practical worry that volatility decreases
guidance remains.

We have our error theory, then. We have identified why one might be
worried by unstable legal norms, even if stability is not an attribute
of legality. The claim that stability is necessary appears reasonable if
one improperly assumes that stability is necessary in order to maintain
the efficacy of the legal system, rather than to facilitate legal
guidance. The core concern is that volatility reduces the strength of
the signal created by law's official mark. Without a strong signal,
ordinary folk will have difficulty locating the content of the law,
which leads to a decline in epistemic guidance, which reduces the
efficacy of the system. This is a serious concern, yet it does not
explain why stability is necessary for law to possess the capacity to
guide conduct. We are thus left with a dilemma.

This dilemma is this: We must reject \emph{either} the claim that
stability is an attribute of guidance \emph{or} the claim that epistemic
guidance is an accurate account of legal guidance. Stability is widely
thought to be an attribute of guidance, and epistemic guidance is the
most prominent conception of legal guidance. Perhaps the way out of this
dilemma is to modify our conception of legal guidance. Thus, we will
consider an alternative conception of legal guidance. On \emph{this}
conception, stability is, in fact, a necessary attribute of guidance.

\textbf{Integrated Guidance}

Integrated Guidance (``IG'') reconceptualizes law's guidance function.
IG treats guidance as possessing both epistemic and motivational
dimensions. In addition to providing epistemic guidance, law motivates
individuals by disposing them to comply with it. IG models individuals
as planners, with the capacity to incorporate legally-sanctioned means
in their plans to accomplish whatever ends they hold. Law guides when
its norms, owing to their status as dispositive settlers of normative
controversies, exert pressure on individuals to use the norms' content
as the bases of plans.\footnote{We find strong hints of IG-like
  positions in the literature, most notably in the work of later Scott
  Shapiro, but also in the work of Postema and Raz.} If the goal of
individual deliberation is to efficiently coordinate a number of
desires, legal norms should outperform any other type of norm, since no
other type of norm promises to be dispositive of controversy.

Broadly speaking, this process can occur via either of two routes:
filtering and facilitating. Legal norms that prohibit certain behavior
filter out unsanctioned means from an individual's plans. Legal norms
that facilitate conduct offer individuals opportunities to employ
sanctioned means to achieve their ends. The content of legal norms
filters out some means, and create others. The result is a legally
sanctioned set of means from which individuals can choose. If
individuals create higher order intentions structured around this
legally sanctioned set of means, then they will be more disposed to
comply with the law. Law influences dispositions by influencing the
structural framework of individual deliberations. If we understand legal
guidance as Integrated Guidance, then a certain level of stability is an
attribute of legality.

Because plans require a certain degree of stability, the means by which
individuals accomplish those plans must also possess a degree of
stability. But we cannot simply leave things there, since individuals
will have different preferences over reconsideration. We will attempt to
sketch how much specificity is required, given a population with certain
preferences over reconsideration.

\textbf{Individuals as Planning Agents}

IG is most naturally modeled around Michael Bratman's planning
conception of individual agency.\footnote{Michael E. Bratman,
  \textsc{Intentions, Plans, and Practical Reason} (Cambridge: Harvard
  University Press) 1987; Shapiro, \emph{supra} note 4.} Individuals
possess certain desires, or outcomes that they hope to achieve. Most of
the time, these desires are not immediately satisfied; certain
intermediate steps need to be taken to render their realization more
likely. The planning conception of agency explains how we structure
means and ends together to help us to achieve our desires. Building on
the planning conception of agency, IG explains how, given the way in
which individuals ordinarily structure their means and ends, legal norms
guide behavior.

Individual action is primarily governed by a set of hierarchical plans,
to which one is non-trivially committed, and by which one is guided in
the making of more specific decisions. The existence of such plans is
essential, because we all face severe resource limitations: limitations
of time, energy, cognitive capacity, and so forth. And, yet, despite
these limitations we nonetheless must coordinate a large set of diverse
activities. Plans are special sorts of individual intentions that help
to overcome these limitations.\footnote{Michael E. Bratman,
  \emph{Planning and the Stability of Intention}, \textsc{Minds and
  Machines}, 2(1992):1--16.} Individuals weave plans of various levels
of generality and complexity together in order to create an agenda for
future action. Rather than lurch from decision to decision, unguided by
any background agenda, individual action is designed to bring about the
goals contained in the plans, and do so efficiently. Inefficient means,
or impossible means, are ``filtered out.''\footnote{Michael E. Bratman,
  David J. Israel, and Martha E. Pollack, \emph{Plans and
  Resource-bounded Practical Reasoning}, \textsc{Computer Intelligence},
  4(1988):350.} Individuals are left with plans that exert strong,
although not absolute, control over any given decision that an
individual makes. Plans can be reconsidered, as we shall shortly see,
but IG treats reconsideration as the exception, not the rule. In case an
individual does reconsider a plan, she might reach for a new plan from
her ``plan library,'' or a set of beliefs about which actions bring
about which effects under a given set of circumstances.\footnote{\emph{Id}.
  For instance, if Alice possesses a plan to travel to a party at 9
  p.m., the she can filter out all options that are incompatible with
  the possibility of attending the party, such as planning a dinner for
  9 p.m. Faced with a smaller set of alternatives, Alice can more easily
  focus on the relevant questions: how to get to the party, and so on.
  And when 9 p.m. arrives, Alice will, in fact, travel to the party,
  absent special circumstances.}

Because plans by their very nature strive to reduce the costs of
accomplishing one's ends, impossible, or inefficient, means are filtered
out. This is the feature of individual agency that allows law to dispose
individuals toward compliance. Subjects of law tend to believe that (a)
purposive agents, namely, legal officials, (b) create legal norms
addressed to subjects (c) in order to settle questions of how to
act.\footnote{Shapiro, \emph{supra} note 12, at 200.} If an individual
has these three well-formed beliefs about law, or even some less
well-formed version of them, legal norms will occupy a highly privileged
place in an individual's deliberations. Because legal systems are not
only designed to but usually effective in settling questions of how to
act, individuals will only accept plans that are consistent with the
ways of acting set forth by legal norms.

Because such agents accept that a hierarchical relationship exists
between themselves and the state, such agents take the content of the
law to provide plans for their own actions and seek to ensure that other
plans are consistent with legal norms.\footnote{\emph{Id}., at 141.} The
content of legal norms thus has the capacity to induce rational agents
to integrate their conduct within the parameters of the law because such
agents \emph{take} the law to be an authority within its domain. Law
guides when its norms, owing to their status as dispositive settlers of
normative controversies, exert pressure on individuals to use their
content as the bases of plans.

Law is designed to settle normative debates. If law does what it claims
to do, it will provide a stable foundation for derivative plans. If the
goals of a deliberative system are to efficiently coordinate a number of
desires, legal norms should outperform any other type of norm, since no
other type of norm guides with the backing of the state. Individuals
thus desire to incorporate its content into their plans. This process of
incorporation can occur via either of two routes: filtering and
facilitating.

\textbf{Filtering and Facilitating}

Legal norms that prohibit certain behavior filter out possible means for
the accomplishment of desires. For instance, if an individual has a
desire to ``get rich quick,'' criminal prohibitions on insider trading
might filter out the means of acquiring material nonpublic
information.\footnote{Such prohibitions may also filter out certain
  ends. One might be enticed by the thought of a life of crime, for
  instance. Our focus on means here should not be read as denying that
  prohibitions also filter ends.} By filtering out certain
possibilities, non-filtered possibilities face fewer competitor means.
Legally sanctioned means are, in a sense, elevated as non-legal means
are winnowed away.

As Hart famously observed, legal norms do more than prohibit conduct;
they create and facilitate new forms of conduct. By making available new
forms of action, legal norms create new means by which individuals can
accomplish their ends. Although it is odd to speak of ``compliance''
with legal norms that facilitate conduct, it is not odd to imagine that
legal officials in some sense favor the use of such ends over the use of
non-legal ends. It makes sense, then, to conceptualize guidance as
disposing individuals to employ legally-created forms of action, even if
the language of compliance is inapposite.

\textbf{Building a Platform of Means}

The content of legal norms filters out some means, and creates others.
The result is a set of legally sanctioned means. If individuals create
higher order intentions structured around this legally sanctioned set of
means, individuals will be more disposed to comply with the law. Law
influences dispositions by influencing the structural framework of --
the higher level plans -- individual deliberations. This dynamic has the
effect of crowding out alternative sources of planning norms. Recall
that legal norms compete against other norms to guide individual
conduct. One of the primary competitive advantages of law is that it
usually has the state's force behind it. Nonetheless, as the fact of
non-compliance shows, legal norms do not always claim victory. IG shows
how legal norms stay competitive by reducing the foothold that
alternatively-sourced norms might have in individuals' planning
mechanisms.

Raz hints at such an idea in his discussion of stability. ``{[}O{]}nly
if the law is stable are people guided by \emph{their knowledge of the
content of the law},'' Raz argues.\footnote{Raz, \emph{supra} note 13,
  at 229 (emphasis in original).} Implicit in Raz's argument is the idea
that, in the absence of legal guidance, individuals will plan according
to alternatively-sourced normative standards. If law is unstable,
individuals are guided by their knowledge of \emph{some other} norm.
Legal norms, though, can crowd out competing norms, and thus make them
less probable to influence individual behavior.

By creating a set of legally sanctioned means, law disposes individuals
toward creating higher order intentions that are compliant with law. The
more that individuals craft their intentions in a legally compliant
manner, the more dependent they become upon legally compliant behaviors.
As with many systems engineered to produce certain outputs, the more
engaged one is with a system, the more dependent one becomes on that
system, the harder it becomes to exit that system. And so it is with
law.

Plans allow individuals to coordinate vast swaths of desires in an
efficient manner. They are efficient solutions in part because they
allow individuals to incur the costs of weighing options, deciding on
means, and coordinating those choices with other of their plans and
other individuals. It thus seems as though plans only efficiently
structure decision-making if they are relatively stable. After all, ``if
we were constantly to be reconsidering the merits of our prior plans
they would be of little use in coordination and in helping us cope with
our resource limitations.''\footnote{Bratman, \emph{supra} note 54, at
  3.} ``Nonreconsideration,'' in other words, ``will typically be the
default,'' even in cases where, were the individual to reconsider the
decision, she would plan differently.\footnote{\emph{Id}.}

Under IG, legal norms must be capable of forming the basis of plans
through legally-sanctioned means. And plans, we just saw, are in the
normal course of things stable. Legal norms, then, too, it would seem,
must be stable. Predicating plans on unstable means seems like a recipe
for incurring the costs of reconsideration. The role of stability under
IG is, at bottom, about the way in which reconsideration affects
individuals' ability to construct plans around legally-sanctioned means.

\textbf{The Costs of Reconsideration}

Plans are premised on the existence of certain conditions. An automobile
maker's plan to build an automobile with \emph{X} level emissions
depends on a regulator's decision that \emph{X} level emissions is
lawful. When the conditions upon which a plan is premised change, an
agent may reconsider her previously formed plans, but she need not.
Non-reconsideration is the default position, but we can sketch with
slightly more specificity the conditions under which individuals
override this default.

Changes in the world exact one of two types of costs in relation to
individual plans. On the one hand, changes can exact costs related to
the agent's deliberative structure. Reconsideration of plans premised on
changes takes time, energy, and inhibit coordination with others. If an
agent prefers to avoid those costs, he might incur a different type of
cost: the costs of attempting to bend a changed world back to a state in
which it is compatible with his original set of plans. For all
individuals interested to keep their plans and the state of the world in
some sense connected, change exacts one or the other of these types of
costs.

Individuals choose different strategies to manage these costs.\footnote{Bratman,
  \emph{supra} note 54.} At one end of the spectrum, individuals' plans
are only as strong as the set of beliefs and desires that formed the
original basis for the plan. If the agent's plans are non-robust in the
face change, then the agent will frequently be attempting to bring his
plans in line with the changed state of the world. The costs typically
borne of such an agent include lost efficiency gains, since he expends
resources recalculating means and ends, and lost coordination gains,
since others that rely on his previous plans will no longer reliably do
so. For instance, consider the deliberative difficulties facing an agent
with a higher order plan grounded in act consequentialism, or some other
set of values that require that an agent constantly calculate the value
of her action. Such plans will struggle to serve the usual purposes of a
plan, since they filter few actions out and ask the agent to continually
deliberate over new options. Nonetheless, some agents may hold such a
view, at least about some matters.

At the other end of the spectrum, plans possess an ``intrinsic
stability,'' which commit the agent to attempt to bring about their
satisfaction conditions, come what may.\footnote{\emph{Id}. at 11.} If
an agent never reconsiders his plans, then volatility will impact the
agent in a characteristic way; namely, the agent will be forced to
attempt to bring the world in line with her (pre-established) plans. The
costs typically borne of this type of agent include those involved with
the open-ended challenges of attempting to make whatever aspect of the
world is no longer compatible with his original set of plans once again
compatible.

Outside of these two poles, individuals tolerate varying levels of
reconsideration. This variance need not be considered the product of
irrationality. Some individuals may be quite adept at shouldering the
costs of reconsideration, while others may be skilled at bending the
world to their wills.

\textbf{Deliberative Types}

We see then, that, individuals' prefer to allocate change costs in
different ways. This allows us to imagine a spectrum of approaches to
the allocation of change costs.\footnote{This section draws heavily on
  Bratman's discussion of stability of intentions. \emph{See} Bratman,
  \emph{supra} note 54.} These have considered two extreme types, but
many positions exist along the spectrum.\footnote{Any given individual
  might be a certain type with regard to one issue, but a different type
  with regard to another issue, since the costs of being one type or the
  other will vary based on the issue and individual capacities.} We will
identify two intermediate types, producing four types in total. We can
label these types as follows:

\begin{quote}
\textbf{1. Non-Robust}: Prior plans are always subject to
reconsideration, given a relevant change.

\textbf{2. Weakly Robust}: Prior plans are subject to reconsideration,
given a relevant change, only if the anticipated costs of
reconsideration are less than the anticipated benefits of
reconsideration, with regard to a specific action.

\textbf{3. Strongly Robust}: Prior plans are subject to reconsideration,
given a relevant change, only if the anticipated costs of
reconsideration are less than the anticipated benefits of
reconsideration, with regard to an agent's overall dispositions about
reconsideration.

\textbf{4. Intrinsically Stable}: Prior plans are never subject to
reconsideration.
\end{quote}

Assume that individuals know their deliberative types, and find it
rational to be that type. As we noted earlier, there are certain costs
characteristic of each type. If an individual knows his deliberative
type, and the costs associated therewith, he will tend to structure his
plans in such a way as to bring about the satisfaction of rational
desires.

The existence of deliberative types complicates IG's account of the
value of planning. Recall that future-direct intentions, or plans, allow
us to deliberate more efficiently and coordinate intra- and
inter-personally.\footnote{Bratman, \emph{supra} note 53.} It would seem
to follow that an agent will not adopt or will tend not to adopt a plan
unless the plan facilitates efficient deliberation and intra- and
inter-personal coordination. Any plan, the content of which is likely
induce inefficient deliberation or inhibit coordination is likely to
remain in an agent's ``plan library'' of possible but unadopted
plans.\footnote{\emph{Id}.}

Given an agent's knowledge of her type, she will rule out some plans in
the plan library on grounds that they are subject to too much
volatility. Given a potential plan from the plan library, there is some
threshold of probability at which the agent, given her tolerance for
reconsideration, will reject the plan. Given the choice of plan
\emph{A}, there is some probability, Pr(\emph{A}), that, out of the set
of possible future events, at least one event that forces the agent to
reconsider will materialize. When that Pr(\emph{A}) reaches the
threshold that the agent sets based on her deliberative type, that
potential plan is rejected.

Strongly Robust and Intrinsically Stable types may be wary of
incorporating plans that they anticipate will be subject to volatility
that destabilizes those plans. This wariness may result from the fact
that they tend not to reconsider plans. Bending the world to one's plans
can be difficult, even where one prefers that course of action to plan
reconsideration. Of course, some changes in the world may induce such
high costs that reconsideration is likely even for Strongly Robust and
Intrinsically Stable types. As a percentage of all changes, however,
these cases are likely to be low, given these types policies about
change.

Strongly Robust and Intrinsically Stable types, precisely because they
make it a policy not to reconsider, may plan for contingencies. If this
is so, then these types may, in fact, \emph{not} be considerably more
wary of potentially unstable plans. Yet, contingency planning is costly,
just as buying insurance is costly. Thus, it will be sought to be
avoided where possible. In general, then, we can hypothesize that
Strongly Robust and Intrinsically Stable types will possess a
comparatively low volatility threshold. In other words, for these
deliberative types, a comparatively low level of volatility will cause
them to leave plans in their planning libraries, rather than adopt them.

For instance, say an Intrinsically Stable type is considering whether to
buy a piece of land on which sits a large lake, for the purposes of
fishing. An assessment of the regulatory climate, however, suggests that
fishing from lakes such as this one might soon be prohibited. The
question is, given our individual's policy of not reconsider plans, our
individual decides not to buy the property. It seem as though, because
of his policy to not reconsider plans, our individual will decide not to
buy the property. But, as an Intrinsically Stable type, he might be
comfortable incurring the risk and attempting to bend the regulatory
process to his preferences. For symmetric reasons, Non-Robust and Weakly
Robust types will possess a comparatively high volatility threshold.

\textbf{Differentiated Guidance}

Law's capacity to guide is typically thought of as a capacity that
extends uniformly over all subjects possessing the capacity to reason.
IG requires a differentiated approach to guidance, since it is premised
on the idea that individuals possess different tolerances for the risks
associated with volatility. By introducing Deliberative Types, we have
introduced differences in the population. We must find some way of
discussing guidance as a capacity possessed by law while also taking
account of these individual differences.

One way in which to handle the problem is to treat law as possessing the
capacity to guide just in case that it does not breach the lowest
rational volatility threshold in the population. Just as epistemic
guidance does not require law to possess the capacity to guide wholly
irrational agents, IG need not require law to possess the capacity to
guide irrationally change-averse agents.

There is a second way to handle the problem of differentiated guidance.
This solution requires analysis of the dynamic and cooperative character
of legal guidance.

Recall that IG argues that the state's goal is to entice individuals to
populate their plans with content derived from legal norms. To guide is
to influence individual dispositions toward legally-sanctioned modes of
conduct. Law influences dispositions by influencing the structural
framework of -- the higher level plans -- individual deliberations.

The state faces a limit on how often it can change the content of these
legal norms, if it desires individuals to use them as part of their
plans. Given a potential plan from the plan library, there is some
threshold of probability at which the agent, given her tolerance for
reconsideration, will reject the plan. This probability threshold,
however, is not permanently fixed or wholly independent of individuals'
wills, as we have thus far imagined.

\textbf{Changing Types}

Whether an individual is of a certain deliberative type is, in some
sense, a function of that individual's desires. If an agent really is
committed to act consequentialism, then it probably is rational to treat
prior plans as always or almost always subject to reconsideration. But
whether an agent is in fact committed to act consequentialism is itself
subject to reconsideration.

Likewise, individuals can seek to reject fewer legal norms on grounds
that they are insufficiently stable. An individual can lower the
probability threshold at which they reject possible legal norms due to
its likely volatility if they move closer to the Non-Robust type.

Even if we hold one's deliberative type constant -- even if one
maintains that an individual's deliberative type is beyond that
individual's capacity to change -- the individual can, nonetheless,
structure his affairs such that the possibility of legal change is less
likely to reach the threshold beyond which the legal norm is perceived
to be too volatile to form the basis of a plan. In other words,
individuals can plan for legal change itself.

But why do so? Purchasing insurance, after all, is costly. Strongly
Robust and Intrinsically Stable types, precisely because they make it a
policy not to reconsider previously established plans, may choose to
plan for certain sorts of contingencies. If this is so, then these types
may \emph{not} be considerably more wary of potentially unstable plans.
However, all things equal, such planning is costly. Why would any
rational individual incur either of these costs in order to be guided by
legal norms?

\textbf{Legal Guidance as Assurance Game}

We explain why individuals incur these costs by observing that the
state's provision of legal norms takes the form of an assurance game.
Both the state and its subjects benefit when the state responds to
widely held problems \emph{and} subjects are disposed to adopt the
regulations. But both of these endeavors are costly, if unreciprocated.
Legal change, then, will tend to only occur when both sets of parties
are sufficiently assured that the other will cooperate.

Let us assume that the state faces a choice between enacting a legal
norm that is responsive to some widespread problem. The state can either
change the content of the law or not change it. Subjects, meanwhile, can
choose to plan for volatility or not plan for it. Of course, some
individuals, namely, those Non-Robust types, need not plan for
volatility in order to cooperate. They are cooperative by their very
type. Outside of the special case of Non-Robust types, however, the
actions of the state and its subjects are interdependent. The state is
better off changing legal norms in response to widespread problem if and
only if subjects tend to plan for volatility.

Likewise, legal subjects are better off incurring the costs of planning
for volatility if and only if the state enacts responsive legal
regulations. Otherwise, they have spent resources -- deliberative and
perhaps material -- preparing for legal guidance that never arrives.

Assuming \emph{a} \textgreater{} \emph{b} \textgreater{} \emph{c}
\textgreater{} \emph{d}, we can represent guidance as an assurance game:

\begin{quote}
\emph{Individuals}

Anticipate Not-Anticipate
\end{quote}

\emph{State} Change

Not-Change

\begin{longtable}[c]{@{}ll@{}}
\toprule
\emph{a},\emph{a} & \emph{d},\emph{c}\tabularnewline
\midrule
\endhead
\emph{c},\emph{d} & \emph{b},\emph{b}\tabularnewline
\bottomrule
\end{longtable}

In order that the state be willing to change a legal norm, it must
anticipate that its subjects will be guided by them. As we observed
earlier, there is some volatility threshold above which individuals of a
certain deliberative type will not incorporate a legal norm into their
planning apparatus. Legal subjects encourage the state to enact new
regulations when they \emph{raise} that threshold.

An unreciprocated attempt at cooperation is worse than maintaining the
status quo. This is true of both the state and its subjects. If subjects
believe that the state cannot or will not enact responsive regulation,
it is preferable not to incur the costs of planning. Likewise, the state
will seek to avoid \emph{attempting} to provide legal guidance if it is
likely to fail. Although such ``defection'' constitutes the second best
set of outcomes for the state and its subjects, all prefer cooperation
to exploitation.

The dynamic conception of stability incorporates the possibility that
individuals will \emph{raise} their volatility thresholds. Although, as
we observed, raising one's volatility threshold is costly, we have now
seen why individuals may be inclined to do so. Individuals cooperate
with the state by planning for beneficial legal change.

If individuals cooperate with the state by planning for beneficial legal
change, volatility thresholds decrease. Perhaps they don't decrease
uniformly, and it's unlikely that they converge on some point, but we
can imagine a something like a general decline and the density around a
certain threshold increases. If this is so, then we have a solution to
our problem of differentiated guidance: volatility thresholds are no
longer so differentiated. If individuals plan for legal change, their
volatility thresholds clump together, thus providing officials with a
target range of stability that must be met.

And, thus, we have derived an answer to the question of why stability is
a necessary condition in order that law guide individual behavior. The
degree of stability necessary in order that law guide individuals is
just that degree required to meet the volatility threshold of some
significant portion of the subject population.

\textbf{Conclusion}

Raz's concern is that instability produces uncertainty, which results in
individuals failing to guide their conduct by law. We have shown that it
is not the instability or the uncertainty \emph{itself} that reduces
guidance; it is, rather, individuals' \emph{expectations} about the
volatility of the norm.

Recall that Raz's claim is that stability requires ``that the making of
particular laws should be guided by open and relatively stable general
rules.''\footnote{Raz, \emph{supra} note 13, at 213.} When this is not
the case, Raz argues, it becomes ``difficult for people to plan ahead on
the basis of their knowledge of the law.''\footnote{\emph{Id}. at 216.}
Our analysis suggests a more complicated picture. Whether individuals
are able to plan ahead on the basis of probabilities about what the law
will be depends on their general deliberative type and the level of
resources that they have devoted to the issue. If the issue matters to
individuals, they will devote resources to understanding the
probabilities that the law will change; they will take out insurance to
compensate against change; and they will seek to push the law in one
direction or the other. What's more, as we have shown, individuals can
recognize that the state's responsiveness to issues can be a public
good, and respond to this realization by planning for legal change.

As one of the primary tools of democracy, law is deployed to respond to
problems. The sorts of problems that contemporary democratic regimes
solve do not arrive in neat intervals; they are often significant and
difficult, and the solutions partial and temporary. Instability is high
because long-term solutions are technically or politically impossible.
To demand stability in the name of legal guidance is to effectively
curtail the process of problem solving.

Because it is preferable that the state be flexible in its approach to
problem solving, it is important to know just how much instability is
compatible with legal guidance. We've tried to move past the simple
refrain that law must be stable, so common in the legal philosophy
literature, and explain just how much stability is, in fact, compatible
with legal guidance. It turns out that the answer to this question turns
on one's conception of guidance. If one subscribes to epistemic
guidance, then stability appears to be inessential to guidance. The
alternative conception we've encountered, Integrated Guidance, does
require some measure of stability. This paper has not adjudicated
between the two conceptions, but we can tentatively conclude that legal
guidance requires less stability than is conventionally believed.
